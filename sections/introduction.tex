\newpage
\section{报告概述}
之所以选择$\lambda$演算, 是因为南京大学软件学院有一门专业课(软件工程与计算I)介绍过函数式编程, 当时授课的老师简单地介绍了函数式编程后的数学背景, 也就是$\lambda$演算. 在大三的数理逻辑课上, 初步了解了命题逻辑和一阶逻辑之后, 我希望从逻辑的角度重新审视我了解过的一些$\lambda$演算理论. 
本次汇报的主要内容参考自Dalhousie University的一份课程讲义\cite{selingerLectureNotesLambda}的一到四章节(主要内容是无类型$\lambda$演算及其编程, Church-Rosser定理), Cambridge University 的一份课程讲义\cite{henkbarendregtLAMBDACALCULUSTYPES}和教材\cite{herbertMathematicalIntroductionLogic}的第零到二章节(主要内容是基础集合论, 命题逻辑, 一阶逻辑)